\documentclass{MetricNotes2023}

\usepackage[a4paper,top=2cm,bottom=2cm,left=2cm,right=2cm,marginparwidth=1.75cm]{geometry}

\usepackage{graphicx}
\graphicspath{ {./images/} }
\usepackage{float}


%inserting a figure:

%\begin{figure}[h]
%\centering
%\includegraphics[width=0.5\textwidth]{pfigure_1}
%\caption{The compound pendulum}
%\end{figure}
%\setcounter{MaxMatrixCols}{20}
\usepackage{amsmath}
\usepackage{enumitem}
\usepackage{amssymb}
\usepackage{amsthm}
\usepackage[colorlinks=true, allcolors=blue]{hyperref}
%\usepackage{hyperref}
\usepackage{mathtools}
\usepackage{parskip}
\usepackage{mathrsfs}

% knots stuff

\usepackage{tikz}
\usetikzlibrary{cd}
\usetikzlibrary{shapes,snakes}
\usepackage{listings}

%\graphicspath{ {figs/} }

\usepackage{cleveref}
\newcommand{\notimplies}{%
  \mathrel{{\ooalign{\hidewidth$\not\phantom{=}$\hidewidth\cr$\implies$}}}}

%keyword: edit

\def\summing{\ensuremath\sum^{\infty}_{n=1}}
\def\summingo{\ensuremath\sum^{\infty}_{n=0}}
\def\bb{\ensuremath\mathbb}
\def\la{\ensuremath\langle}
\def\ra{\ensuremath\rangle}
\def\notto{\ensuremath\nrightarrow}
\def\subq{\ensuremath\subseteq}

\def\met{\ensuremath(X, d)}
\def\metx{\ensuremath(X, d_X)}
\def\mety{\ensuremath(Y, d_Y)}
\def\topox{\ensuremath(X, \mathcal{T}_X)}
\def\topoy{\ensuremath(Y, \mathcal{T}_Y)}

\def\comp{\ensuremath\mathbb{C}}
\def\real{\ensuremath\mathbb{R}}
\def\rat{\ensuremath\mathbb{Q}}
\def\inte{\ensuremath\mathbb{Z}}
\def\nat{\ensuremath\mathbb{N}}
\def\topo{\ensuremath\mathcal{T}}

\def\A{\ensuremath\mathcal{\mathscr{A}_2}}

\def\sequiv{\ensuremath\stackrel{\text{S}}{\sim}}
\def\rr{\ensuremath\mathcal{R}}
\def\com{\ensuremath\Delta}
\def\id{\ensuremath\text{id}}
\def\hopf{\ensuremath\mathcal{H}}
\def\psip{\ensuremath\psi_{+}}
\def\psin{\ensuremath\psi_{-}}
\def\psipm{\ensuremath\psi_{\pm}}
\DeclareMathOperator{\colim}{colim}
\DeclareMathOperator{\trace}{trace}
\DeclareMathOperator{\tr}{tr}
\DeclareMathOperator{\str}{str}
\DeclareMathOperator{\End}{End}
\DeclareMathOperator{\Ext}{Ext}
\DeclareMathOperator{\Hom}{Hom}
\DeclareMathOperator{\lk}{lk}
\DeclareMathOperator{\im}{im}

\def\done{\begin{flushright}\vspace{-4.35ex}\(\qed\)\end{flushright}}

\def\mlip{\ensuremath\abs f_\text{Lip}}


\DeclareSymbolFont{bbold}{U}{bbold}{m}{n}
\DeclareSymbolFontAlphabet{\mathbbold}{bbold}

% \limits\sum or \sum\limits (one of them) will put the text under the sum
% [upquote=true] on \begin{lstlisting} will stop backticks being interpreted as quotes
%[backend=bibtex, style=ieee]
\usepackage[backend=bibtex]{biblatex}
%\bibliographystyle{plain}
%\usepackage{csquotes}
\addbibresource{References.bib}

\author{\vspace{-5ex}}
\title{Stable Homotopy Groups of Spheres [DRAFT]}
\date{\vspace{-5ex}}

\counterwithin{figure}{section}



\begin{document}
\maketitle
%\input{titlepage.tex}

\DeclarePairedDelimiter{\norm}{\lVert}{\rVert} 
\DeclarePairedDelimiter{\abs}{\lvert}{\rvert} 
\DeclarePairedDelimiter{\ang}{\langle}{\rangle} 

\tableofcontents

\pagebreak

\section{Introduction}

\begin{itemize}
\item Define homotopy groups
\item The Eilenberg-MacLane space is \(K(G, n)\), and it has the property that 
\[\pi_i(K(G, n))=\begin{cases}
\inte & i=n,\\
0 & i\neq n.
\end{cases}\]
They're unique up to weak homotopy equivalence (i.e. if you have another one \(X\), there's a map between them which descends to an isomorphism on homotopy groups)
\item Define suspension of a topological space
\item Freudenthal's suspension theorem: if \(\pi_i(X)=0\) for \(i\leq k\) (i.e. \(X\) is \(k\)-connected) then the map 
\begin{align*}
\pi_n(X) \;\;&\to\;\; \pi_{n+1}(\Sigma X)\\
[\gamma : S^n \to X] &\mapsto [\Sigma \gamma : \Sigma S^n=S^{n+1} \to \Sigma X]
\end{align*}
is an isomorphism for \(n \leq 2k\) and surjective for \(n=2k+1\)
\item This implies \(\pi_{n+k}(S^n)\) depends only on \(k\) for \(n\geq k+2\)
\item (Obviously be careful with basepoints above)
\item Suppose \(X\) is \(k\)-connected. Then, for \(k\geq 0\), \(0=\pi_k(X)\cong \pi_{k+1}(\Sigma X)\), so whenever a space is \(k\)-connected its suspension is \(k+1\)-connected. 
\item As you take suspensions, then, your successive bounds are \(n \leq 2k\), \(n+1\leq 2k+2\implies n \leq 2k+1\), \(n\leq 2k+2\), etc ... so the sequence \(\pi_n(X)\to \pi_{n+1}(\Sigma X)\to \cdots\) will eventually stabilise.  
\item Thus, if you take the colimit of that direct system, it'll just equal the stable value, with the higher legs just being the inverse isomorphisms.
\item \autocite{ass}, Cor 1.9 [not 100\% convinced of how this follows, but believing it for now]: if \(X\) is a CW complex of dimension \(d\) and \(Y\) a \((k-1)\)-connected space, then the suspension homomorphism \([X, Y]\to[\Sigma X, \Sigma Y]\) is bijective if \(d<2k-1\) and surjective if \(d=2k-1\). 
\end{itemize}

Miscellaneous facts I might need later:
\begin{itemize}
\item Cohomology [possibly only of pointed CW complexes] is representable, and its representing object is the Eilenberg-MacLane space.  i.e. \(H^n(-; G)\cong \Hom(-, K(G, n))\). 
\item There is an adjunction \(\Sigma \dashv \Omega\), where \(\Omega\) is the loop functor.
\item \(\mathscr A_2\) is generated as an algebra by elements \(Sq^{2^k}\) (\autocite{hatcher}, Prop 4L.8).
\item The map \(\mathscr A_2 \to \tilde H^*(K(\inte/2\inte, n); \inte/2\inte)\), \(Sq^I\mapsto Sq^I(\iota_n)\) is an isomorphism from the degree \(d\) part of \(\mathscr A_2\) onto \(H^{n+d}(K(\inte/2\inte, n); \inte/2\inte)\) for \(d \geq n\). In particular, the admissible monomials \(Sq^I\) form an additive basis for \(\mathscr A_2\). Thus, \(\mathscr A_2\) is exactly the algebra of all \(\inte/2\inte\) cohomology operations that are stable, commuting with suspension (\autocite{hatcher5}, Cor 5.38). 
\item ``Stable homotopy groups are a homology theory'' (whatever that means)
\item Hurewicz theorem: for any path-connected space \(X\) and \(n>0\) there exists a group homomorphism \(h_* : \pi_n(X)\to H_n(X)\). For \(n=1\) this induces an isomorphism \(\pi_1^{\text{ab}}(X)\cong H_1(X)\). For \(n \geq 2\), if \(X\) is \((n-1)\)-connected then \(\tilde H_i(X)=0\) for all \(i<n\), and the map \(h_* : \pi_n(X)\to H_n(X)\) is an isomorphism. 
\end{itemize}

Algebraic background:

\begin{definition}
Let \(M, N\) be modules over a ring \(R\). A \textit{free resolution} \(F\) of \(M\) is an exact sequence 
\[\cdots \to F_2 \to F_1 \to F_0 \to M \to 0,\]
with each \(F_i\) a free \(R\)-module.
\end{definition}

Applying \(\Hom_R(-, N)\) gives us a chain complex
\[\cdots \leftarrow \Hom_R(F_2, N) \leftarrow \Hom_R(F_1, N) \leftarrow \Hom_R(F_0, N) \leftarrow \Hom_R(M, N) \leftarrow 0.\]
Dropping the term \(\Hom_R(M, N)\) [why?] we get the sequence
\[\cdots \leftarrow \Hom_R(F_2, N) \leftarrow \Hom_R(F_1, N) \leftarrow \Hom_R(F_0, N) \leftarrow 0,\]
and we define \(\Ext^n_R(M, N)\) to be the \(n\)th homology group of this chain complex. 

%we write \(H^n(F; G):=\ker f^*_{n+1}/\im f_n^*\). Any abelian group \(H\) has a free resolution of the form 
%\[0 \to F_1 \to F_0 \to H \to 0\]
%(the one you think it is). So we get a chain complex
%\[0 \leftarrow F_1^* \leftarrow F_0^* \leftarrow H^* \leftarrow 0.\]
%We have \(H^n(F; G)=0\) for \(n>1\). Define \(\Ext(H;G):=H^1(F;G)\). 

[these do not depend on the choice of free resolution of \(M\)]

\autocite{ass}, \autocite{suspension}, \autocite{hatcher}

\section{The Steenrod algebra}

The following is from \autocite{hatcher} 4L.

\begin{itemize}
\item There are maps \(Sq^i : H^n(-; \inte/2\inte)\to H^{n+i}(- ; \inte/2\inte)\) for each \(i\), and they satisfy the following properties: \begin{enumerate}
\item \(Sq^i_X(f^*(\alpha))=f^*(Sq^i_Y(\alpha))\) for \(f : X \to Y\) (i.e. \(Sq^i\) is a natural transformation).
\item \(Sq^i_X(\alpha + \beta)=Sq^i_X(\alpha)+Sq^i_X(\beta)\) (i.e. \(Sq_X^i\) respects the group operation for all \(X\)).
\item \(Sq^i(\alpha \smile \beta)=\sum\limits_{0\leq j \leq i} (Sq^j(\alpha)\smile Sq^{i-j}(\beta))\) (the Cartan formula)
\item \(Sq^i(\sigma(\alpha))=\sigma(Sq^i(\alpha))\) where \(\sigma : H^n(X; \inte/2\inte)\to H^{n+1}(\Sigma X; \inte/2\inte)\) is the ``suspension isomorphism given by reduced cross product with a generator of \(H^1(S^1; \inte/2\inte)\)''
\item \(Sq^i(\alpha)=\alpha^2\) if \(i=\abs{\alpha}\) and \(Sq^i(\alpha)=0\) if \(i> \abs{\alpha}\). [Hatcher doesn't explain this notation at all, but I think he means by \(\abs{\alpha}\) the degree of \(\alpha\) - this is what \autocite{stable_homotopy2} says in C2]
\item \(Sq^0=\id.\)
\item \(Sq^1\) is the ``\(\inte/2\inte\) Bockstein homomorphism \(\beta\) associated with the coefficient sequence \(0 \to \inte/2\inte \to \inte/4\inte \to \inte/2\inte \to 0\)''. 
\end{enumerate}
\item Define \(Sq:=Sq^0+Sq^1+\cdots\). Then \(Sq(\alpha\smile \beta)=Sq(\alpha)\smile Sq(\beta)\) (since \((Sq(\alpha\smile \beta))_n=\sum_iSq^i(\alpha)\smile Sq^{n-i}(\beta)=(Sq(\alpha)\smile Sq(\beta))_n\)). Thus, \(Sq\) is a ring homomorphism. 
\item Adem relations:
\[Sq^aSq^b=\sum_j {b-j-1\choose a-2j}Sq^{a+b-j}Sq^j \quad \text{if } a<2b,\]
where \({m \choose n}\) is zero if \(m\) or \(n\) is negative, or \(m<n\), and \({m \choose 0}=1\) for \(m \geq 0\).
\item The Steenrod algebra \(\mathscr{A}_2\) is the algebra over \(\inte/2\inte\) that is the quotient of the algebra of polynomials in the noncommuting variables \(Sq^1, Sq^2, ...\) by the two-sided ideal generated by the Adem relations. Thus, for every space \(X\), \(H^*(X; \inte/2\inte)\) is a module over \(\mathscr A_2\), via \(\alpha \cdot f = f(\alpha)\).
\item \(\mathscr A_2\) is graded, and its elements of degree \(k\) are those that map \(H^n(X; \inte/2\inte)\) to \(H^{n_k}(X, \inte/2\inte)\) for all \(n\). [Presumably you've fixed a space \(X\) while you're doing all this?]
\end{itemize}

\autocite{stable_homotopy}, \autocite{cobordism}, 
\autocite{ass}, \autocite{spectra}, \autocite{hatcher}, \autocite{foundations}

\section{Spectra may not be your friends, but I can introduce you}

\begin{itemize}
\item \autocite{ass}: There is a category \(\mathcal{H}\) of finite [because the corollary wanted f.d. CW complexes] based CW  complexes, with \(\Hom(X, Y)=:[X, Y]\) the set of homotopy classes of base-point preserving maps \(X\to Y\).
\item There is a category \textbf{St}(\(\mathcal{H}\)) of finite[?] based CW complexes, with \(\Hom(X, Y)=:\{X, Y\}\) the  set \(\colim_i [\Sigma^iX, \Sigma^iY]\) [it's just a colimit of sets, and \textbf{Set} is cocomplete, so we should be fine. \autocite{ass} says it's a group?] 
\item There is a functor \(\mathcal{H}\to \textbf{St}(\mathcal{H})\). \autocite{ass} doesn't say what this is but it's presumably the one that is the identity on objects and sends \([f : X \to Y]\in [\Sigma^0X, \Sigma^0Y]\) to whatever it gets sent to in \(\{X, Y\}\) using the universal property of the colimit. Uniqueness makes it functorial, etc.
\item We have a fully faithful functor \(\textbf{St}(\mathcal{H})\to \textbf{St}(\mathcal{H})\) given by the suspension on objects, and the unique isomorphism \(\{X, Y\}\to\{\Sigma X, \Sigma Y\}\) on maps (such an isomorphism exists, since both of those things are colimits for \([\Sigma^i X, \Sigma^i Y]\) - one of the sequences is cut off at the beginning, but it doesn't matter because both reach the stable value (see above discussion and \autocite{ass} 1.9), aka the colimit). 
\item It's not an equivalence, because not every object is isomorphic to a suspension (e.g. anything not connected, since suspensions always connected [?])
\item We can formally adjoin desuspensions \(\Sigma^{-n}X\) for all \(n\) [does this mean just putting the objects there and defining \(\Hom(Y, \Sigma^{-n}X):=\Hom(\Sigma^nY, X)\) and \(\Hom(\Sigma^{-n}X, Y):=\Hom(X, \Sigma^n Y)\)?], but this category does not have weak colimits (i.e. colimits w/o uniqueness property). [why does it not, and why do we even want that?]
\item We instead consider formal sequences of desuspensions \(X_0 \to \Sigma^{-1}X_1 \to \cdots\), or sequences \((X_n)\) and maps \(\Sigma X_n \to X_{n+1}\), i.e. spectra. [and this fixes the problem?]
\end{itemize}

\begin{definition}
A \textit{spectrum} is a collection of pointed topological spaces \(\{X_n\}_{n\in \nat}\), together with basepoint-preserving maps \(\sigma_n : \Sigma X_n \to X_{n+1}\).
\end{definition}

\begin{example}
Let \(X\) be a topological space. The \textit{suspension spectrum} of \(X\) has \(X_n=\Sigma^nX\) and \(\sigma_n=\id : \Sigma X_n \to X_{n+1}\).
\end{example}

[Define EM spectrum]

\begin{definition}
Let \(X=\{X_n\}\) be a spectrum. We define \(\pi_i(X)=\colim_n \pi_{i+n}(X_n)\), where the map \(\pi_{i+n}(X_n)\to \pi_{i+n+1}(X_{n+1})\) is given by the composition
\[\pi_{i+n}(X_n)\xrightarrow{\Sigma}\pi_{i+n+1}(\Sigma X_n)\xrightarrow{(\sigma_n)_*}\pi_{i+n+1}(X_{n+1}).\]
\end{definition}

\begin{example}
If \(X'\) is the suspension spectrum of a topological space \(X\), then \(\pi_i(X')\) is the \(i\)th stable homotopy group of \(X\). 
\end{example}

\autocite{stable_homotopy}, \autocite{cobordism}, \autocite{ass}, \autocite{spectra}, \autocite{foundations}

\section{The Adams spectral sequence}

\autocite{spectral_sequences}, \autocite{stable_homotopy}, \autocite{cobordism}, \autocite{foundations}

\section{\(\Ext_A^s(\bb{F}_2, \bb{F}_2)_t\)}

\autocite{stable_homotopy}, \autocite{cobordism}, \autocite{ass}

\section{Methods of resolving ambiguities}

\autocite{stable_homotopy}, \autocite{cobordism}

\printbibliography

\end{document}